\documentclass[UTF8, 10pt, a4paper, oneside]{ctexart}
\usepackage{amsmath}
\usepackage{amsthm}
\usepackage{amsfonts}
\usepackage{amssymb}
\usepackage{amstext}
\usepackage[version=4]{mhchem}
\usepackage{geometry}
\usepackage{changepage}
\usepackage{paralist}
\usepackage{graphicx}
\usepackage{extarrows}
\usepackage{xcolor}
\geometry{left=1.27cm, right=1.27cm, top=1.27cm, bottom=1.5cm}
\linespread{1.5}
\title{\vspace{-2em} 高考模拟卷 \vspace{-1em}}
\author{盛炯元 \quad 杨元谦}
\date{\textcolor{white}{\today}\vspace{-3em}}
\pagestyle{plain}

\newcommand{\blank}{ \underbar{\quad$\blacktriangle$\quad} }% 空格样式
\newcommand{\fs}[1]{{\fangsong #1}}% 使用仿宋
\newcommand{\circled}[1]{{\small{\textcircled{\tiny{#1}}}}}% 圈圈数字
\newcommand{\Romannumeral}[1]{\uppercase\expandafter{\romannumeral#1}}% 大写罗马数字
\newcommand{\chdots}{…\hspace{-0.15em}…}% 中文省略号调教

\theoremstyle{definition}
\newtheorem{exercise}{}
\newtheorem{subexercise}{}[exercise]% 用于兼容编号错误或是内含的高考题等

\theoremstyle{remark}
\newtheorem*{answer}{【答案】}
\newtheorem*{point}{【考点】}      % 考点&易错点
\newtheorem*{method}{【方法】}     %(可选)
\newtheorem*{explanation}{【解析】}     %

\theoremstyle{plain}
\newtheorem*{note}{【注】}  %(可选)

\begin{document}
\maketitle
\begin{exercise}
    在硝化细菌中,不会发生的生命活动是\quad(\quad)
    \begin{center}
        \begin{inparaenum}[\qquad A. ]
            \item[\setcounter{enumi}{1}A. ] 核膜的消失与重建
            \item 核酸-蛋白质复合物的形成与消失
            \item 氧气的摄取
            \item ATP的合成与水解
        \end{inparaenum}
    \end{center}
    \begin{answer}
        A
    \end{answer}
    \begin{point}
        原核与真核细胞的差异、化能合成作用
    \end{point}
    \begin{explanation}
        原核生物是原核细胞,没有以核膜为界限的细胞核,不存在增殖时周期性的核膜消失与重建,A选项错误;在基因转录、复制时,会有RNA聚合酶或DNA聚合酶与DNA分子结合形成核酸-蛋白质复合物,B选项正确;在化能合成作用中,硝化细菌需要借助氧气氧化氨或亚硝酸盐,C选项正确;ATP作为能量货币,能量在各生化反应中流动伴随着ATP的合成与水解,D选项正确。
    \end{explanation}
\end{exercise}
\begin{exercise}
    蛋白质是结构和功能多样的生物大分子,下列叙述正确的是\quad(\quad)
    \begin{adjustwidth}{2em}{}
        \begin{asparaenum}[A. ]
            \item 二硫键的断裂不会改变蛋白质的空间结构
            \item 血浆里,人体三大供能物质中质量分数最大的是蛋白质
            \item 向蛋白质溶液中加入浓的硫酸铜溶液可使蛋白质发生盐析
            \item 利用基因工程可以“创造”出自然界中原先不存在的蛋白质
        \end{asparaenum}
    \end{adjustwidth}
    \begin{answer}
        B
    \end{answer}
    \begin{point}
        蛋白质空间结构、血液中糖脂蛋相对含量、盐析与变性、基因工程局限性
    \end{point}
    \begin{explanation}
        二硫键影响蛋白质的三、四级结构,进而影响其整体空间结构,A选项错误;血浆中水占91\%$\sim$92\%,蛋白质约占7\%\textsuperscript{\fs{[义务教育教科书·生物学七年级下册 P35]}},B选项正确;铜离子是重金属,会使蛋白质发生变性,而不是盐析,C选项错误;基因工程原则上只能生产自然界中已存在的蛋白质,D选项错误。
    \end{explanation}
\end{exercise}

\begin{exercise}
    在一定温度下,大田的某种植物绿色叶片的光合速率和呼吸速率的平均对光照强度的响应曲线如图所示。下列说法错误的是\quad(\quad)
    \begin{adjustwidth}{2em}{}\vspace{-3em}
        \begin{figure}[h!]
            \flushright
            \includegraphics[width=0.3\textwidth]{assists/2-1.jpg}
        \end{figure}\vspace{-15em}
        \begin{asparaenum}[A. ]
            \item 生长于光照强度为a环境的植株生长最将因匮乏营养而死亡
            \item 光照强度从a逐渐增加到b时,该植物生长速率逐渐增大
            \item 光照强度小于b时,提高大田$\ce{CO2}$浓度,$\ce{CO2}$固定速率会增大
            \item 光照强度为b时,适当降低光反应速率,$\ce{CO2}$固定速率会降低
        \end{asparaenum}\vspace{0.5em}
    \end{adjustwidth}
    \begin{answer}
        C
    \end{answer}\vspace{0.25em}
    \begin{point}
        总光合速率、净光合速率\vspace{0.5em}
    \end{point}
    \begin{explanation}
        生长于光照强度为a环境的植株虽然其叶片的总光合速率与呼吸速率相等,但植株中总有无法进行光合作用的组织(如根、花、果实)消耗着贮存的有机物,A选项正确;光照强度小于b时,影响光合作用的主要因素为光照,提高大田$\ce{CO2}$浓度不一定h使$\ce{CO2}$固定速率增大,C选项错误。
    \end{explanation}
\end{exercise}

\begin{exercise}
    葡萄糖转运蛋白4(GLUT4)是一种存在于脂肪细胞中的蛋白质。在胰岛素的刺激下,GLUT4会从脂肪细胞内的囊泡膜上转移至细胞膜上,葡萄糖借助细胞膜上的GLUT4进入脂肪细胞。下列叙述错误的是 \quad(\quad)
    \begin{adjustwidth}{2em}{}
        \begin{asparaenum}[A. ]
            \item 脂肪细胞中GLUT4以氨基酸为原料,在核糖体中合成
            \item GLUT4转移至细胞膜所需要的能量主要来自于线粒体
            \item GLUT4每次转运葡萄糖时,其自身构象都会发生改变
            \item 当血糖浓度升高时,脂肪细胞膜上的GLUT4数量减少
        \end{asparaenum}
    \end{adjustwidth}
    \begin{answer}
        D
    \end{answer}
    \begin{point}
        蛋白质的生物合成、主动运输、胞间信息传递
    \end{point}
    \begin{explanation}
        当血糖浓度升高时,胰岛B细胞受刺激分泌更多胰岛素,促进更多GLUT4从脂肪细胞内的囊泡膜上转移至细胞膜上,故D选项错误。
    \end{explanation}
\end{exercise}





\begin{exercise}
    某研究团队测定了水分胁迫(不足或过多)下黄芩叶片的生理指标,结果如表所示。下列叙述正确的是\quad(\quad)
    \begin{figure}[h!]
        \centering
        \includegraphics[width=0.8\textwidth]{assists/4-1.jpg}
    \end{figure}\vspace{-1em}

    (注:在胁迫状态下,细胞积累的活性氧会破坏膜的脂质分子,形成丙二醛)
    \begin{adjustwidth}{2em}{}
        \begin{asparaenum}[A. ]
            \item 氧自由基会攻击膜的组分蛋白质分子时,产物也是自由基,引发雪崩式反应
            \item 水分过多时,细胞膜完整性受损,控制物质进出的能力减弱
            \item 水分不足时,细胞内可溶性糖含量的变化导致水分流失增加
            \item 无论水分供给过多或不足,叶片内叶绿素含量相同,推测叶绿素的合成与水分供给无关
        \end{asparaenum}
    \end{adjustwidth}
    \begin{answer}
        B
    \end{answer}
    \begin{point}
        细胞衰老——自由基学说、数据分析
    \end{point}
    \begin{explanation}
        自由基攻击蛋白质分子会使蛋白质活性下降,但产物不是自由基,当自由基攻击磷脂时,产物才也是自由基,引发雪崩式反应,A选项错误;注意到水分过多时,细胞内可溶性糖含量显著降低,控制物质进出的能力减弱,B选项正确;水分不足时,胞内可溶性糖含量升高,渗透压升高,更不易流失水分,C选项错误;无论水分供给过多或不足,叶片内叶绿素含量都减少,叶绿素合成与水分供给有关,D选项错误。
    \end{explanation}
\end{exercise}
\begin{exercise}
    生菜是一种在保存和运输过程中易发生褐变的蔬菜。多酚氧化酶(PPO)在有氧条件下能催化酚类物质形成褐色的醌类物质,导致植物组织褐变。某团队研究生菜多酚氧化酶在不同条件下的特性,结果如图所示。下列叙述正确的是\quad(\quad)
    \begin{figure}[h!]
        \centering
        \includegraphics[width=0.8\textwidth]{assists/6-1.jpg}
    \end{figure}
    (注:吸光度大小与醌类物质含量成正相关)\vspace{-1em}
    \begin{adjustwidth}{2em}{}
        \begin{asparaenum}[A. ]
            \item 低氧环境可促进褐变发生
            \item 柠檬酸有抗氧化作用
            \item 保存和运输生菜的最适pH值为6
            \item 40$^\circ$C时PP0活性最高,适于生菜保存
        \end{asparaenum}
    \end{adjustwidth}
    \begin{answer}
        B
    \end{answer}
    \begin{point}
        数据分析
    \end{point}
    \begin{explanation}
        褐变的实质是氧化,故低氧环境能减缓褐变的发生,A选项错误;随着柠檬酸浓度上升,多酚氧化酶的活性下降,褐变(氧化)减缓,故柠檬酸有抗氧化作用,B选项正确;当pH为6时,吸光度最高,褐变(氧化)程度最高,不宜用于保存和运输,C选项错误;当反应温度为40$^\circ$C时,吸光度最高,褐变(氧化)程度最高,不宜用于保存和运输,D选项错误。
    \end{explanation}
\end{exercise}
\begin{exercise}
    生长于NaCl浓度稳定在100 mmol/L的液体培养基中的酵母菌,可通过离子通道吸收$\ce{Na+}$,但细胞质基质中$\ce{Na+}$浓度超过30 mmol/L时会导致酵母菌死亡。为避免细胞质基质中$\ce{Na+}$浓度过高,液泡膜上的蛋白N可将$\ce{Na+}$以主动运输的方式转运到液泡中,细胞膜上的蛋白W也可将$\ce{Na+}$排出细胞。下列叙述错误的是\quad(\quad)
    \begin{adjustwidth}{2em}{}
        \begin{asparaenum}[A. ]
            \item $\ce{Na+}$在液泡中的积累有利于酵母细胞吸水
            \item 蛋白N转运$\ce{Na+}$过程中自身构象会发生改变
            \item 通过蛋白W外排$\ce{Na+}$的的过程不需要细胞提供能量
            \item $\ce{Na+}$通过离子通道进入细胞时不需要与通道蛋白结合
        \end{asparaenum}
    \end{adjustwidth}
    \begin{answer}
        C
    \end{answer}
    \begin{point}
        主动运输、被动运输、渗透作用
    \end{point}
    \begin{explanation}
        生长在NaCl浓度稳定在100 mmol/L的液体培养基中,而细胞质基质中$\ce{Na+}$浓度超过30 mmol/L时会导致酵母菌死亡,可以推知通过蛋白W外排$\ce{Na+}$的的过程是逆浓度梯度的主动运输,需要细胞提供能量,C选项错误。
    \end{explanation}
\end{exercise}
\begin{exercise}
    淀粉类种子萌发形成幼苗离不开糖类等能源物质,也离不开水和无机盐。下列叙述正确的是\quad(\quad)
    \begin{adjustwidth}{2em}{}
        \begin{asparaenum}[A. ]
            \item 种子吸收的水与多糖等物质结合后,水仍具有溶解性
            \item 种子萌发过程中糖类含量逐渐下降,有机物种类不变
            \item 幼苗根细胞中的无机盐可参与细胞构建,水不参与
            \item 幼苗中的水可参与形成NADPH,也可参与形成NADH
        \end{asparaenum}
    \end{adjustwidth}
    \begin{answer}
        D
    \end{answer}
    \begin{point}
        结合水、种子的萌发
    \end{point}
    \begin{explanation}
        结合水失去了流动性和溶解性,成为生物体的构成部分,A选项错误;种子萌发过程中糖类含量逐渐下降,有机物种类增多,B选项错误;结合水是细胞结构的重要组成部分,也参与细胞的构建,C选项错误;幼苗进行光合作用时,光反应阶段中水光解产生的 $\ce{H^+}$ 和被叶绿体夺去的电子与 $\ce{NADP+}$ 结合生成 NADPH,幼苗进行有氧呼吸时,在第二阶段丙酮酸和水分解成 $\ce{CO2}$ 和 NADH,D选项正确。
    \end{explanation}
\end{exercise}
\begin{exercise}
浆细胞合成抗体分子时,先合成的一段肽链(信号肽)与细胞质中的信号识别颗粒(SRP)结合,肽链合成暂时停止。待SRP与内质网上SRP受体结合后,核糖体附着到内质网膜上,将已合成的多肽链经SRP受体内的通道送入内质网腔,继续翻译直至完成整个多肽链的合成并分泌到细胞外。下列叙述正确的是\quad(\quad)
    \begin{adjustwidth}{2em}{}
            \begin{asparaenum}[A. ]
                \item SRP与信号肽的识别与结合具有特异性
                \item SRP受体缺陷的细胞无法合成多肽链
                \item 核糖体和内质网之间通过囊泡转移多肽链
                \item 生长激素和性激素均通过此途径合成并分泌
            \end{asparaenum}
    \end{adjustwidth}
    \begin{answer}
        A
    \end{answer}
    \begin{point}
        point
    \end{point}
    \begin{explanation}
        即使SRP受体缺陷,浆细胞仍能合成信号肽这一多肽链(举反例),B选项错误;核糖体和内质网依赖于SRP与SRP受体的高亲和力结合,不依赖于囊泡,C选项错误;性激素是固醇,不是蛋白质,不通过上述路径合成或分泌,D选项错误。
    \end{explanation}
\end{exercise}

\end{document}