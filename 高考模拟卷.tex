\documentclass[UTF8, 10pt, a4paper, oneside]{ctexart}
\usepackage{amsmath}
\usepackage{amsthm}
\usepackage{amsfonts}
\usepackage{amssymb}
\usepackage{amstext}
\usepackage[version=4]{mhchem}% 规范:chemfig中键线式放缩为0.5;结构简式放缩为0.6;高分子链节\polymerdelim[delimiters={[]},height=4pt, depth=4pt, indice=n]{left}
\usepackage{chemfig}
\setcharge{shortcuts=true}
\usepackage{geometry}
\usepackage{changepage}
\usepackage{paralist}
\usepackage{multicol}
\usepackage{graphicx}
\usepackage{polyglossia}
\usepackage{extarrows}
\setotherlanguages{russian}
\newfontfamily\russianfont{Times New Roman}
\geometry{left=1.27cm, right=1.27cm, top=1.27cm, bottom=1.5cm}
\linespread{1.5}
\title{\vspace{-2em} 高考模拟卷 }
\author{盛炯元 \quad 杨元谦}
\date{\textcolor{white}{\today}\vspace{-3em}}
\pagestyle{plain}

\newcommand{\blank}{ \underbar{\quad$\blacktriangle$\quad} }% 空格样式
\newcommand{\fs}[1]{{\fangsong #1}}% 使用仿宋
\newcommand{\circled}[1]{{\small{\textcircled{\tiny{#1}}}}}% 圈圈数字
\newcommand{\Romannumeral}[1]{\uppercase\expandafter{\romannumeral#1}}% 大写罗马数字
\newcommand{\chdots}{…\hspace{-0.15em}…}% 中文省略号调教

\theoremstyle{definition}
\newtheorem{exercise}{}
\newtheorem{subexercise}{}[exercise]% 用于兼容编号错误或是内含的高考题等

\theoremstyle{remark}
\newtheorem*{answer}{【答案】}
\newtheorem*{point}{【考点】}      % 考点&易错点
\newtheorem*{method}{【方法】}     %(可选)
\newtheorem*{explanation}{【解析】}     %

\theoremstyle{plain}
\newtheorem*{note}{【注】}  %(可选)

\begin{document}
\maketitle

\begin{exercise}
蛋白质是结构和功能多样的生物大分子,下列叙述正确的是\quad(\quad)
\begin{adjustwidth}{4em}{}
        \begin{asparaenum}[A. ]
            \item 二硫键的断裂不会改变蛋白质的空间结构
            \item 血浆里,人体三大供能物质中质量分数最大的是蛋白质
            \item 向蛋白质溶液中加入浓的硫酸铜溶液可使蛋白质发生盐析
            \item 利用基因工程可以“创造”出自然界中原先不存在的蛋白质
        \end{asparaenum}
\end{adjustwidth}
\begin{answer}
    B
\end{answer}
\begin{point}
    蛋白质空间结构、血液中糖脂蛋相对含量、盐析与变性、基因工程局限性
\end{point}
\begin{explanation}
    二硫键影响蛋白质的三、四级结构,进而影响其整体空间结构,A选项错误;血浆中水占91\%~92\%,蛋白质约占7\%\fs{[义务教育教科书·生物学七年级下册 P35]},B选项正确;铜离子是重金属,会使蛋白质发生变性,而不是盐析,C选项错误;基因工程原则上只能生产自然界中已存在的蛋白质,D选项错误。
\end{explanation}
\end{exercise}

\end{document}